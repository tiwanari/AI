\documentclass{jarticle}
\usepackage{mathtools, multicol}
\usepackage{color}
\usepackage{url}
\usepackage{comment}
\usepackage{here}
\usepackage{txfonts}
\usepackage{listings, jlisting}
\usepackage{latexsym}
\usepackage{subfigure}

\renewcommand{\lstlistingname}{リスト}

\lstdefinestyle{customplain}{
  belowcaptionskip=1\baselineskip,
  breaklines=true,
  frame=tRBl,
  xleftmargin=\parindent,
  language=,
  showstringspaces=false,
  numbers=left,
  basicstyle=\footnotesize\ttfamily,
  keywordstyle=\bfseries\color{black},
  commentstyle=\itshape\color{black},
  identifierstyle=\color{black},
  stringstyle=\color{black},
}

% 余白の設定
\usepackage[top=20truemm, bottom=16truemm, left=10truemm, right=10truemm]{geometry}

% 図の挿入
\usepackage[dvipdfm]{graphicx}

% より複雑な数学記号
\usepackage{amsmath,amssymb}

% 図の通し番号
\usepackage{subfigure}

\newcommand{\todayd}{%
\the\year.{\ifnum \month < 10 0\the\month \else \the\month \fi}.%
{\ifnum \day < 10 0\the\day \else \the\day \fi}}


\makeatletter

\def\@thesis{人工知能}
\def\id#1{\def\@id{#1}}
\def\department#1{\def\@department{#1}}

\def\@maketitle{
	\begin{center}
		{\huge \@thesis \par} %大きなタイトルが記載される部分
		\vspace{10mm}
		{\LARGE\bf \@title \par} % タイトル部分
		\vspace{20mm}
		{\Large 提出締切: 2014.01.09\par} % 提出年月日部分
		\vspace{5mm}
		{\Large 提出日:  \@date \par} % 提出年月日部分
		\vspace{20mm}
		{\Large \@department \par} % 所属部分
		\vspace{10mm}

		{\Large\@id } % 学籍番号部分
		{\Large \@author} % 氏名 
	\end{center}
\par\vskip 1.5em
}

\makeatother

\title{第7回講義課題 課題番号20}
\date{\todayd}
\department{工学部電子情報工学科}
\id{03-123006}
\author{岩成達哉}


\begin{document}

\begin{titlepage}
	\setlength{\topmargin}{1.1in}
	\vspace{100mm}
	\maketitle
\end{titlepage}

\begin{comment}
ACO(アントコロニー・アルゴリズム)を 応用し、Nクィーン問題を解くプログラムを作成せよ。   
各種パラメータ(ρ:フェロモンの蒸発率、もしあればα:TSPでの距離の重み指数に相当するパラメータなど)
問題サイズ
を様々に変化させ、実験結果を比較・考察すること。
以下の論文などが参考になる。

S. Khan et al.“Solution of n-Queen problem using ACO”. In proc. of 13th IEEE International Multi-topic Conference (INMIC 2009),2009.
\end{comment}


\section{概要}
本レポートでは,アントコロニー最適化(ACO)を用いて,Nクイーン問題を解くプログラムを作成し,各種パラメータに依る探索の時間やステップの変化を調べた.



% ----
\section{ACOアルゴリズム}

\subsection{概要}
ACOアルゴリズムは,アリがホルモンを分泌することで,他のアリに道を示すという習性になぞらえたアルゴリズムである.アリは道の中で最も良い物に強くホルモンを残し,他のアリはそのホルモンを元に道を知ることができる.また,ときにはそのホルモンに従わないアリが現れることで,別の道を模索することができる.

この習性を探索アルゴリズムとして用いることで,良い探索経路を多くたどるようになり,かつ,ときには新しい探索経路を調べることで,他に良い解がないか探すことができる.

\subsection{手順}
Nクイーン問題を解くACOとして,参考文献\cite{ref:aco}がある.この手法では,以下の手順で探索が進む.
\begin{enumerate}
	\item フェロモンの初期化を行う.フェロモンは$N^2\text{[移動元]} \times N^2 \text{[移動先]} $個だけ用意する. 
	\item $N^2$個のマスに対して,アリを複数匹ランダムに設置する.このとき,同じマスに複数匹置いても良い.
	\item アリを一匹ずつフェロモンを元に$N^2$個のマスに移動させる.このとき,今までいたことのあるマスには置けない.$i$番目のマスから$j$番目のマスに移動する確率は以下のようになる.ただし,$S$は今までいたことのないマスの集合,$\tau_{i,j}$は$i$番目のマスから$j$番目のマスへのフェロモン,$\zeta_{i,j}$は$i$番目のマスから$j$番目のマスへ行くヒューリスティックな評価値である.
		\begin{eqnarray}
			p_{i,j} = \frac{[\tau_{i,j}]^\alpha \cdot [\zeta_{i,j}]^\beta}{\sum_{k \in S} [\tau_{i,k}]^\alpha \cdot [\zeta_{i,k}]^\beta}
		\end{eqnarray}
		ヒューリスティックな評価値は,Nクイーン問題では例えば,「次にそのマスにアリを移動させることによってクイーンがいくつ死ぬか」などを与えることができる.
	\label{enu:loop_ant}
	\item \ref{enu:loop_ant}を$N-1$回繰り返す.
	\item 一匹のアリが移動したマスを盤面に投影したものが一つのパターンとなる.全てのアリの経路に対して,解が得られているかを調べる.解でなかった場合は,そのアリが通過した経路すべてのフェロモンを更新する.Nクイーンでは例えば,アリが$i$番目のマスから$j$番目のマスへ移動していた場合,殺しあったクイーンの数を$L$,ある定数を$Q$とすると,
		\begin{equation}
			pheromone[i][j] \leftarrow pheromone[i][j] + \frac{Q}{L}
		\end{equation}
	と更新できる.
	\label{enu:update_pheromone}
	\item 全てのフェロモンを蒸発させる.ある定数$\rho$を使って,
		\begin{equation}
			pheromone[i][j] \leftarrow pheromone[i][j] \times \rho
		\end{equation}
		とあらわせる.これは,\ref{enu:update_pheromone}によってフェロモンが増え続けるのを防ぐ役割がある.
\end{enumerate}




% ---
\section{プログラムの実行方法}
プログラムはC言語で作成した.ソースコードのコンパイルは,Makefileによって行うことができる.具体的には,コマンドプロンプトを用いてソースコードのあるフォルダに移動してリスト\ref{code:make}のようにコマンドを実行すれば良い.
\lstset{style=customplain}
\begin{lstlisting}[caption={make},label=code:make]
$ make
\end{lstlisting}

実行は,リスト\ref{code:execute}のように行う.Nには正の整数を与える.これが,Nクイーン問題の問題の大きさとなる.
\begin{lstlisting}[caption={実行},label=code:execute]
$ ./aco N
\end{lstlisting}

実行されると,Nクイーン問題の一つの解を探索し,結果を示す.解がない場合は無限ループとなるため,収束しない場合を含めて$1000$回のステップで終了するようにしている.



% ---
\section{実験方法}
実験は,学科PC(Ubuntu 12.04 [Intel Core i7 @2.40GHz,8コア,メモリ 16GB]) によって行った.

\subsection{$\rho$を変化させたときのステップ数}
蒸発率$\rho$を変化させ,そのステップ数の変化をみた.実験方法は以下のとおりである.
\begin{enumerate}
	\item 
	\item 
\end{enumerate}


% --
\section{結論}
本課題では,ACOアルゴリズムを用いて,Nクイーン問題の解を求めるプログラムを作成し,このアルゴリズムが確かに正しい解を求めること,パラメータによってその探索速度が異なることが確かめられた.




\begin{thebibliography}{n}
\bibitem{ref:aco}
S. Khan et al.“Solution of n-Queen problem using ACO”. In proc. of 13th IEEE International Multi-topic Conference (INMIC 2009),2009.,\url{http://www.iba.t.u-tokyo.ac.jp/~iba/AI/khan.pdf}


\end{thebibliography}


\appendix
\section{ソースリスト}

ソースのリストを以下に示した.

\begin{itemize}
	\item Makefile \\
		makeをするためのファイル
	\item util.c / util.h \\
		汎用的に用いる関数群.メモリの確保などを含んでいる.
	\item aco.c / aco.h \\
		Nクイーン問題を解く,ACOアルゴリズムを実装したもの.
	\item main.c \\
		ACOアルゴリズムを呼び出すメイン関数.問題の大きさを入力するなどの機能を持つ..
\end{itemize}

\end{document}













