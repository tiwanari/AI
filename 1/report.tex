\documentclass{jarticle}
\usepackage{mathtools, multicol}
\usepackage{color}
\usepackage{url}
\usepackage{comment}
\usepackage{here}

% 余白の設定
\usepackage[top=20truemm, bottom=16truemm, left=10truemm, right=10truemm]{geometry}

% 図の挿入
\usepackage[dvipdfm]{graphicx}

% より複雑な数学記号
\usepackage{amsmath,amssymb}

% 図の通し番号
\usepackage{subfigure}

\newcommand{\todayd}{%
\the\year.{\ifnum \month < 10 0\the\month \else \the\month \fi}.%
{\ifnum \day < 10 0\the\day \else \the\day \fi}}


\makeatletter

\def\@thesis{人工知能}
\def\id#1{\def\@id{#1}}
\def\department#1{\def\@department{#1}}

\def\@maketitle{
	\begin{center}
		{\huge \@thesis \par} %大きなタイトルが記載される部分
		\vspace{10mm}
		{\LARGE\bf \@title \par} % タイトル部分
		\vspace{20mm}
		{\Large 提出締切: 2013.10.31\par} % 提出年月日部分
		\vspace{5mm}
		{\Large 提出日:  \@date \par} % 提出年月日部分
		\vspace{20mm}
		{\Large \@department \par} % 所属部分
		\vspace{10mm}

		{\Large\@id } % 学籍番号部分
		{\Large \@author} % 氏名 
	\end{center}
\par\vskip 1.5em
}

\makeatother

\title{第1回講義課題 課題番号01}
\date{\todayd}
\department{工学部電子情報工学科}
\id{03-123006}
\author{岩成達哉}


\begin{document}

\begin{titlepage}
	\setlength{\topmargin}{1.1in}
	\vspace{100mm}
	\maketitle
\end{titlepage}


\section{題材}
「ミンスキー博士の脳の探検 ―常識・感情・自己とは―」(共立出版)を題材とした.


\section{要約,主題,主旨}
本書は,人工知能を作るために人間の脳を学び,人工知能を作ることを試みることで人間の脳を学ぶという2つの目的を持った本である.この本の特徴は,今まで脳の研究において実施されてきた「脳の働きを説明するできるだけ簡易な法則を発見する」というアプローチが脳の理解の妨げになってきたと述べ,むしろ「複雑な説明を求める」ことを念頭においていることである.





%---
序論では,人間の思考を考えるために,人間の脳を非常に多くの構成部品を含んだものとして扱うことを提案する.これは,たくさんの思考の部品となる要素 - 思考素 - が脳にあり,このうちの幾つかを活性化,あるいはその反対に抑制している状態が一つひとつの思考であるという理論である.このとき,「悲しみ」と「怒り」など別の思考が同じ思考素を活性化する場合もある.すなわち,思考素は個々で機能するものではなく,その相互作用によって様々な思考を表すわけである.これが,人間が多様な感情を混在させることのできる理由でもある.

さらに,現在のAIと人間の最も大きな違いとして,ある問題に対し様々な解法やアプローチが作れることを挙げている.もし,同様の問題に対して一つの方法でしか理解できないのであれば,それは実際には理解できているとはいえない.なぜなら,その方法が失敗した時にどうすることもできないからだ.このことは,人間とAIだけでなく,人間と他の生物とを隔てる壁ともなっている.

なぜ人間にこのようなことができるのか,AIと人間はなにが違うのか.心を持ったAIを作るためには,この多様性を備えなければならないと述べる.このことを考えるために,人間の脳を探求することがこの本の主題である.





%---
%人は多くの心的思考素とともに生まれてくる.
第1章では,人間の心を知るために様々な問題を取り上げる.これらの問題一つひとつを各々の章で少しずつ見ていき,後半に向かうにしたがって理解を深めていくのである.

まず,愛とは何かということから始まる.愛や感情という言葉は,本来は様々な意味を持つ多くの言葉を一つの入れ物 - スーツケース - に詰め込んでいる.このような言葉を細分化できないものとして扱えば,「愛とはなにか」という質問に対してそれ以上追求する必要がなくなる.つまり,実用上の観点から,従来の心の研究と同様に,思考を「できるだけ簡単な説明で」表そうというのである.実際は愛というスーツケースに閉じ込められた部品を知ることが本質的であるにも関わらず.ここで,筆者は「初めは単純に見える心的事象について,より複雑に説明をする」という探求をしていく.その一つが,「恋」という『スーツケースワード』の仕組みを前述の思考素で説明することであり,恋は相手の批判する思考素を抑制した状態として述べるのである.また,「単一自己」というテーマも取り上げる.

さらに,人間の行動を形作る最も簡単なモデルとして「If $\to$ Doルール」を取り上げる.人間は生まれながらにして,多くのIf $\to$ Doルールをもって生まれてくる.例えば,「暑ければ日陰に移動する」などである.本書では,このルールの単純さが,説明を簡単にすることを示す.しかし,このような単純な法則では,人間の多様な思考を説明できない.そこで,より高次のモデルは,思考素を基に説明する.

高次のモデルは,人間の思考がどのように成長するのかという問題と関わってくる.人間が思考をするためには,見たものを行動に影響させるように,低次のプロセスから高次のプロセスに情報の伝達をする必要がある.この際に,高次のプロセスは,低次のプロセスが提供した情報から行動を選択することしかできない.この役割分担を『批評家』と『選択家』という言葉で表現する.さらに高次の思考については,これらに加えて人間の心が階層で表されるようになるとして扱う.





%---
%人は他人と交流することから,より多くを学ぶ
2章では,1章で取り上げた思考の成長のプロセスについて,子供がどのように学習するかということから考察を始める.そして,子供の学習には周囲の人間が大きく影響することを述べる.

人間はあることをする際に目標があり,その目標を達成するために必要な下位目標がある.例えば,スープを飲む場合を考える.このとき,人間がフォークを使ってスープをすくおうとすると失敗するとわかる.次に,スプーンを使うとうまくすくえるとわかる.このようにして,子供は試行錯誤の上に,スープを飲むこととスプーンを使うことを関連付ける.

なぜスプーンをスープを飲むことと関連付けることができるのだろうか.他の要因が間違って関連付けられることはないのだろうか.例えば,そのときにしかめっ面をしていたらしかめっ面がスープを飲むことと関連付けられないのだろうか.実はそれ自体が学習なのである.言い換えれば,学習とは,「目標と下位目標を関連付けること」ではなく,「目標と下位目標を関連付ける構造を作ること」なのである.

また,試行錯誤の他に,学習をする際にインプリマ(imprinting[刷り込み]からの造語)が関連付けを手助けする場合がある.インプリマとは,子供が愛着を持つようになった人のことを意味し,インプリマが褒めたり,叱ることで,目標と下位目標の関連付けが強化される.筆者は,褒めることや叱ることで影響を受けるということは,そのときの子供の感情が学習に大きく寄与することを意味すると述べる.

では,なぜインプリマのようなプロセスが学習には必要なのであろうか.まだどの要素が目標達成に影響するか判断できない幼少期では,すでに判断のできるほど成長した大人が介入することが良い方法だと考えられるだろう.しかし,見知らぬ人によって,子供の目標が操作できてしまうと,その人によって子供は意のままに操られてしまう.したがって,正しい影響を与えてくれると考えられるインプリマが,幼少期の学習に必須の要素として位置づけられたのである.

そして,このように学習を重ねていくと,いずれ子供は自分の中にインプリマのモデルを持つことになり,一人で内省できるようになる.そして,インプリマから自立した学習が行えるようになるのである.





%---
%感情は,私達が利用する様々な思考路である.
3章では,悲痛や苦痛などの感情が大きなテーマとなっている.痛みは,人間が生きていくために必須のものである.痛みがなければ危険を感じることができない.しかし,人間は痛みに感謝するのではなく,不満を漏らすことが多い.人間にとって,痛みとは不快なものであり,取り除くべきであると認識しているからである.

人間が痛みを表現するとき,他の痛みで表現することが多い.これは,感覚を説明することが非常に難しいからである.感覚は,心的状態の全体を表現するものだと筆者は述べる.心の状態は複雑で,どんなに言葉を尽くしても,その側面しか捉えられない.そこで,我々は他の状態で例えることで表現し,類推をするしかないのである.そして,我々は精神的な苦しみと身体的な苦しみをその例えとして双方に利用する.

身体的な痛みと比較して精神的な痛みには長続きするものがある.これも,人間が進化する上で,危険から体を守るように発達した機能であるが,精神的に苦痛を感じると他のことは全く考えられなくなることも多々である.なぜ人間はこんなにも苦しむようにできているのだろうか.これは実は,単純な「プログラムのバグ」であると筆者は述べる.豊かな思考を持つまでは,痛みから逃れることを優先して進化することが必要であったため,内生的な思考や長期的計画とは両立しない仕組みが成立してしまったのである.このような,バグは間違った決断を下す元にもなる.

しかし,人間はよく失敗するにも関わらず,それが致命的な結果を招くことは少ない.なぜなら,人間には,「批評家」という複数の思考素を持っているからである.これらの思考素は,「すべきではないことを知っているエキスパート」であり,間違ったことをしないか様々な段階で監視している.これを前述のIf $\to$ Doルールに適用すると,批評家は現在の状況を認識するIfの役割を担っている.そして,Doに対応するのが状況に基づいてある思考素を活性化する「選択家」である.この選択家は,2章で述べたような学習に基づいて,類推によって似たような状況で役立った思考素を活性化する役割を担う.また,この選択のために,批評家は行動を抑制するだけでなく,経験から効果が大きいものを奨励するということも行う.つまり,上位の選択家は,下位の批評家によって大きく操作されるのである.そして,If $\to$ Doよりもずっと高次の思考でも,この批評家と選択家が豊かに作用し合うのである.





% ---
%自身の直近の思考について考察する
4章では,「意識」というスーツケースワードを紐解いていく.著者は意識を,本能的なものから理性的なものまでの6つの階層を持った構造で表現する.人間は,「今自分が意識していることを意識すること」はできない.それは,人間の脳はそもそも別の領域に全くと言っていいほど手出しができないもので,介入できたとしてもその時点でその状態を変えてしまうからである.

しかし,我々は何を考えているかについては考えなくても様々なことを成し遂げられる.例えば,脳がどのようなメカニズムで働き,身体のどの細胞を活性化させるのか知らなくても,音楽を聞いたり,ものを運んだりできる.このような能力は「意識」によってなされるものであるがゆえに,我々は意識を神秘的なものとして捉える.実際のところ,「意識」という言葉はあまりに多くのものを含んでおり,理解するのが非常に難しいこともその一因となっている.

意識の表現方法は様々で現在のところ正解と呼べるものはない.心のあるマシンを作ろうというのであれば,意識を構成する非常に多くのものを一つひとつ探求することが必要であり,マシンを「定義する」ことではなく,「設計を試みる」ことが重要なのである.





% ---
%複数の階層の中で考察することを学ぶ.
5章では,前述の意識の階層について,なぜ筆者がそれらの階層を仮定したのかが詳しく述べられる.つまり,「本能的反応」,「学習反応」,「熟考」,「内省的思考」,「自己内省的思考」,「自意識の内省」の6つの階層を一つひとつ詳しく見ていくのである.

ここでは,If $\to$ Doルールでは十分ではないとし,それを拡張したIf $\to$ Do $\to$ ThenというようにThen(結果)を含めたモデルを考えることで説明を行う.このモデルでは,Ifから順に結果を見ていくとその組み合わせは膨大になってしまうことが容易に想像できる.これに対して,筆者は,人間がIfとThenの両端から内側に向かって考えることで考える組み合わせを減らしていると説明する.

さて,この階層モデルにも問題がある.それは,これらの階層の境界が曖昧であることであり,著者もその自覚を持っている.しかし,まだよくわかっていないものに対しては,一見して欠点に思われるが「考えるべき場所を残す」ことが必要であり,このようなモデルで考えておくことが良いと筆者は述べる.そして,複雑な心は複数のモデルを使って表すことをさけられないのである.例えば,著者はこれらの階層をフロイトの超自我,自我,イドとも結びつける.

この章のもう一つのテーマは,この多階層のプロセスがどのように働くかということである.階層の働きを知るために,筆者らは,Builderという積み木を自動で行うロボットを作ることで実証した.その答えとして,Builderでは積み木を認識するために様々な手法を階層的に用いたが,結局のところそれらの手法を組み合わせ,相互に作用させることが最も良い方法であった.階層は,低次から高次へのバケツリレーではなく,高次が低次に影響を与えることも許されるのである.





% ----
%膨大な常識的知識を蓄積する
6章では,人間が多くの「常識」を持って生きていることをテーマとする.現在のマシンは,ある目的のために作られ,それ以外のことに対する知識は持っていない.例えば,銀行システムで桁を間違えて誤発注しても,マシンはそれが間違いか判断する知識を持っていないのである.また,もっと言えば,今日のプログラムは自らの目標を認識していない.指示された通りのことを行なっているだけなのである.ロボットが経験から学べない理由はここにある.

ここでは,新しい概念として『パナロジー』という言葉がでてくる.これはある出来事を複数の領域で対応付ける構造のことであり,例えば,「本を渡す」という動作には,「物理的に本を手渡す」という解釈や「所有権を渡す」という解釈が存在する.このように異なった様々な領域で同じものを対応付けることが,ある問題へのアプローチを多様にすることに繋がる.このときに必要な物も経験から蓄えられた常識的知識であり,人間は問題に対面したとき,その常識から類推を使って同じ対処法をとったり,あるいは差分を考えて対応することができるのである.

この章では,ロボットがどのようにして知識を習得したら良いかを『赤ちゃんマシン』というアイディアを使って考える.このマシンは,人間と同じように初めはほとんど何も知らない状態から始めて,少しづつ学習して成長するものである.しかし,このような幾つもの取り組みは必ず失敗してきた.それは,「どのような知識を学習すべきか」という問題が解決できなかったからである.この要因の一つが,知識を表現出来なかったことが挙げられる.表現できないことは学べないわけだ.

また,どの知識を取り入れるかという問題もある.我々の記憶の概念は「その物事が達成に役立ちそうなものか」からなる.しかし,未だに現状のAIでは,「それがどのように記述されているか」という分類しかできていない.そして,間違った知識を習得したときの副作用を抑える方法も十分でない.マシンが学習する有力な方法としてあげられるのは『差分エンジン』である.これは,ある目標と現況を比較し,その差がなくなるように状況を変えていくというものである.つまり,目的を持つことが重要なのである.





%---
%さまざまな思考素の中でスイッチを切り替える
7章は,思考についてより深く考える.人間の際立った特徴とは,新しい思考路を形成できることである.ある問題について思考をする際には,前述の批評家$-$選択家モデルが適用できる.批評家は問題のタイプを判定し,ある選択家を活性化する.選択家は,経験に基づいて問題解決に役立ちそうな思考素群を起動しようとする.このようにわざわざ2段階のプロセスを経る理由は,先祖が選択家のような高次の機能を必要としなかったからであると仮定するのが自然であろう.そして,このことは同じ批評家が別の選択家を起動できるという多様性にもつながる.この章では,どのような思考路や批評家が有用であるかということにも踏み込む.さらに,感情と思考の関係を考察する.結論としては,感情は思考路の一つとみなせる.感情は身体にも影響を及ぼすが,その理由は,感情を身体に表すことでその効果を持続させ,感情を保つという身体と感情のフィードバック・ループを作成することが有用だからである.

筆者は,この章でアンリ・ポアンカレなどの偉人を題材に,無意識に依る思考について考察する.人間は,あることを考える際に,無数の組み合わせを無意識に試している.全くその問題を考えていないのに,ふとしたきっかけで突然解法がひらめくというのは,無意識のプロセスに依るものである.そして,そのプロセスは,十分に考えた中で最善と思われるものを提示し,評価することができる.ある人が「わたしは自由意志で解決方法を選択した」と述べることは,「あるプロセスがわたしの熟考を止めて,そのときに最善と思われる選択をさせた」ということに等しいのである.

さて,このように思考について調べる際に,統計心理学を用いることを筆者は否定している.なぜなら,思考は個々によって異なるものであり,統計の結果は,ある影響と別の影響が打ち消し合って,データとして主張しなくなるということが多々起こるからである.そして,批評家が判定する主要な問題タイプとはなにか,選択家が好んで取る選択肢はなにか,これらの思考素を管理する高次の組織はなにかという3つの分類を深く研究することが必要だと述べる.





%---
%物事を表現する複数の方法を見出す
8章では,今まで考察してきた内容を多面的に追求し,思考の豊かさについて考える.コンピュータは一つのことを一つの方法でしかできない.しかし,人間は様々なアプローチを考えて,その中で選択することができる.例えば,椅子の大きさやものの距離を様々な方法で認識する.このために,近くの物は遠く見えるということや,近くのものは早く動いて見えること,濃淡が光のあたり具合を表すなどの膨大な常識を備えている.そして,脳が考える方法を切り替える際には,前述のパナロジーを利用する.

さらに,人間の学習がなぜこれほどまでに早いのかということも考える.犬に芸を教える場合,犬は何回も練習を重ねなければ芸を習得できない,しかし,人間はその芸を一度見て教えられるよう.この理由は,人間が内省的に前述の差分エンジンを使って「訓練師」を育てるからである.このように自分で内省できるということが,学習を手助けしている.また,学習には「起因の特定」が必要となる.つまり,成功した理由は何か,失敗した理由は何かということである.人間はこの起因の特定が他の生物に比べて優れていることが最も重要な思考能力としてあげられる.いわゆる天才や偉人はこのような学習能力が優れているのである.

また,この章では,知識を表現する方法を複数取り上げる.そして,それらの方法は全て長所と短所を持っている.では,どの表現方法が正しいのであろうか.答えは,「それらの方法を組み合わせる」ことである.これによって,長所と短所を補う,階層構造の表現方法を筆者は提案する.結局,人間の思考の豊かさとは,同じ状況を記述するために複数の方法を用いることに起因しているのである.





%---
%複数の自己モデルを構築する
9章では,いよいよ「自己」というものを考える.AIは自己を持てるのか,そもそも自己とはなにか.我々は,自分の中にいる「小人」として自己を表現する.つまり,小人が目や耳などから信号を受け取り,動作を行うまでの操作をしていることにして,心的プロセスの働きを問うことをやめるのである.しかし,この論理は明らかに破綻している.なぜなら,その小人の自己は,さらに内部の小人が関わっており,その小人の中の小人の中にも,自己としての小人がいるというように終わりのない論理となっているためだ.筆者は,意識について,階層やネットワークなど幾つもの例を挙げ,脳を説明しようとした.しかし,それぞれに長所と短所があり,どれが良いということはなかった.これは,もっと良い表現があるという言うよりも,統一的に扱おうという試みが不可能であるということを示しているのではないかと問う.むしろ多面的に研究することが理解につながるのである.つまり,「自己が単一である」というものは神話に過ぎないのである.

筆者は,今までの研究では,2つに分けることが重要視されすぎていたと述べる.2つにのみ分類すればよいでのあれば確かに考えることが簡単になる.しかし,0か1かの判定では分けきれないものがあり,2つに分けることで本質を見分けることを困難にする場合がある.例えば,人間の脳について,右脳はより直感的な考え方,左脳はより論理的な考え方をするように役割が分担されていると言われている.しかし,近年の研究で,2つの脳が同じ機能を持っていることがわかってきた.実は,その違いは,右脳よりも左脳が発達することで生まれるものであり,右脳が引き起こす本能的な活動を,左脳の理性が抑えるのである.そこで筆者は,心理状態を考えるときは2つに分けるではなく必ず3つ以上に分けてることから始めるべきだと提言する.

さらに,この章では,人間の脳は経験することを喜ぶことに触れ,ネガティブな経験がむしろ原因の特定に手をかしているという論理を展開する.そして,その感情について,なぜ表現が難しいのか,人間の心はどのように成り立っているかのまとめを行う.生物は想像の出来ないような時間を過ごして進化してきた.人間の脳は,今まで死んだ生物のたくさんの実験によって成り立っているのである.その中には,もちろんバグがあり,脳の大半は他の部分の不具合を抑えるために働いている.逆に言えば,脳は他の部分の不具合による影響を修正する方法として進化してきたのである.過去には必要のあった機能が必要なくなり,悪影響を与えている部分もある.

以上のように,この本では人間の脳をじっくりと深く探検してきた.このことを活かし,AIを作ることを自分たちでもっと経験することが必要である.そして,それは同時に人間の脳の研究にもつながる.AIを研究することは,未来の人類の心を知るのにつながってくるだろうと締めくくる.





%-----
\section{考察及び感想,疑問に思った点,理解しがたい点}
この本を読んで,様々な新しいアイディアに驚愕したというのが素直な感想である.AIを作ることは難しいのだろうとなんとなく思っていた私にとって,なぜ難しいのか,どうしたらよいのかということを学ぶ手引としては最高の本だったと思う.しかし,著者が提言するように,難しく説明することを意識した本であったため,すんなりと理解することは難しかった.

本書では,人間の脳を探求することでAIに活かし,AIを設計することで人間の脳を知るという相互に利益のある目的を目指している.そこで,人間の脳について考えてみると,作者が「思考素」と呼ぶ,非常に多くの要素で成り立っていると考えられる.そして,それらの要素は1つひとつでは意味をなさず,相互に作用しあうことで思考を行うということを学んだ.授業では,「中国人の部屋」という中国語を知らない人が辞書を使ってTuring Testを突破できるという問題について,様々な議論があると教わった.ここでの最も重要な論点は「中国語を理解しているかどうか」であり,その中で「人と辞書をひっくるめたシステム全体としては中国語を理解している」という話があったが,まさに人間は膨大な細胞が集まったシステム全体として機能している.したがって,私はこのモデルが中国語を理解していると見なして良いと考えている.

私は,AIを作る際には,6章でまさに扱われた「赤ちゃんマシン」のようなものを作れば良いのではないかと考えていた.しかし,そのような試みがなぜ失敗するのか考えたこともなかったのだ.現在のAIの目指すべき姿は,学習すべきことを表現し,選択することであると知り,さらにそれには膨大な量の常識が必要であると本書に教えられた.授業で扱ったバッテリーを洞窟に取りに行って,爆弾も一緒に持ってきて爆発してしまうロボットの話はまさに,必要な常識が足りていないのである.

このことに,納得はしたものの,そのことは「現在の科学力では強いAIは作れない」ということを意味しているのではないかと感じた.その理由の一つに,人間のように振る舞える強いAIを作るためには,一つの仕事を遂行する弱いAIと異なり,目的を限定されない様々な常識を蓄えるための膨大な記憶領域が必要であるからということが挙げられる.また,人間のように感じ,表現するための目や鼻に相当するセンサやアームなどがなければ,人間と同じような学習をこなすことは出来ない.少なくとも移動できるほどの質量の記憶領域では,人間と同じようには振る舞えないであろう.

また,本書では,AIが学習するためにどの記憶が役立つかを考えるということが大きな課題であると学んだ.しかし,私は,この点に関して,「どの記憶を捨ててもよいか」ということについての議論がなされていないように感じた.目的を達するために必要な下位目標を結びつける構造を学習する必要があるのはわかるが,結び付けられる下位目標が複数あっても良いのではないだろうか.その一方で,結びつけなくても良いもの,あるいは結びつけてはいけないものを考える必要もあると考える.

さらに,これはAIには出来ないのではないかと感じたものが,7章の「無意識のプロセス」である.人間は考えを一旦やめておき,無意識のうちに膨大な組み合わせを試して最善手を導く.AIはこのように,自分で考えていることを「意識しないで」思考を巡らせることができるのであろうか.今までのAIでは,意識して思考をする領域(AIが自分が考えていることがわかるという意味の意識)については試行錯誤をされてきたが,無意識に思考をする領域については考えてこられなかったように思う.AIが人間のように無意識のうちに別の思考をするということは,並列処理をすることにあたるのだろうか.並列処理は,AIが意識せずに行っているといえるのだろうか.私がもっとも疑問に思うことは,AIが人間のように振る舞うようにするために,どこまで人間と同じ思考のプロセスをもたせるべきなのかということである.

人間の脳は,バグをたくさん抱えおり,脳の大部分は他の部分の不具合を抑えるために働いていると本書では何度も述べられる.人間の脳を学ぶことでAIを作ることは,このバグも取り入れることになるのであろうか.本書では,脳を研究することでAIを学び,AIを研究することで脳を学ぶとするが,人間がもつバグをいくつか除いたAIを作ることができるはずである.しかし,それが未来の人類の心を知ることにつながるとは思えない.なぜなら,そのバグの幾つかは取り除けないものも含んでいるに違いないからである.AIは「人間がこのようなバグをもっていなかったらどうなるか」という研究につながるものでもあるのではないだろうか.

本書を通して人間の脳の構造を学んだわけだが,人間自身も必ずしも理性的な生物ではない.時に,理論とは異なった行動や思考をする.AIを作る際は,このことをどうするべきなのだろうか.私は,この仕組みはAIに取り入れたほうが良いのではないかと思う.論理に従ったほうがプログラムとして組む際には実用的であることは間違いない.しかし,その論理で行き詰まり,自己矛盾が起こるような場合も日常には存在する.言い換えるならば,論理的でない思考を取り入れることで人間は自分を守っているのではないだろうか.そして,そのことは思考の多様性にもつながっていると考える.

さて,この本で考察することが面白いだろうと感じたものに,「AIに目標を持たせる」という項目がある.現在のAIは,自分で目標を持って行動するのではなく,プログラムされたように仕事をこなすだけである.これによって,目標のためにより良い他の可能性があっても,目標を認識していないAIはそれらの手法を取ることはない.人間が「閃く」ことができるのは,ある明確な目標を持っていて,その目標のために様々な手法を「使えるものは使う」という精神で望んだ結果ないだろうか.こう考えると,AIに目標を持たせないことには,大きな進歩を得ることはできない.しかし,「AIに目標を持たせる」とはどういう意味なのだろうか.

例えば,条件を設けて「誤差がある値以下になるまで近似の精度を高めろ」とプログラムを書くことは目標を持っていることにはならないのであろうか.私は,これは目標を持っていることになると考える.ただし,AIは目標を持っても,1つの手法しか使わないのであれば意味が無く,目標を理解しているとはいえない.これは序論でも述べられたことである.実際は,このような簡単な例とは異なり,我々の生活では目標を表現することが難しいものが多いということも問題になってくる.さらに,目標を表現することが難しいだけでなく,AIが「自ら目標を持つ」ということができていないことが議論すべき対象であると考える.

そもそも我々は目標を持たなければ生きていけない.何も目標がないのであれば,生きている意味が無いからである.それでも生きているのは,「死なない」ことが目標となっているからかもしれない.目標は十人十色である.では,AIは何を目標にするのだろうか.なぜAIは生まれてくるのだろうか.ある目標があったとき,そのことを達成するためにどのような機構をもたせれば良いかを考えてきたが,そもそもAIが自ら持つ目標はあるのだろうか.自ら目標を持てないのであれば,人間のように学習することはないのではないと私は考える.この議論をもっとすべきだと思う.




\begin{thebibliography}{n}
\bibitem{ref:book}
Marvin Minsky (原著), 竹林 洋一 (翻訳) :『ミンスキー博士の脳の探検 ―常識・感情・自己とは―』,共立出版 (2009.7.10)

\end{thebibliography}

\end{document}












